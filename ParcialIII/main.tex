\documentclass[10pt,a4paper]{article}
\usepackage[T1]{fontenc}
\usepackage{graphicx}
\usepackage{mathtools}
\usepackage{amssymb}
\usepackage{amsthm}
\usepackage{hyperref}
\usepackage{sagetex}


\begin{document}
	
	
	\title{Estabilidad de la ecuación de onda}
	\author{Thomas Gomez, Antonio Vargas}
	\date{\today}
	\maketitle

\textbf{Solución:}

La ecuación de onda para una dimensión se reduce a:

\begin{equation}
\frac{\partial^2 u}{\partial^2} = \alpha^2 \frac{\partial^2 u}{\partial^2	}
\end{equation}

La estabilidad de una ecuación diferencial se refiere, suponga ahora que el método se resuelve a través del método de diferencias finitas, la solución con diferencias finitas es:

\begin{equation}
u_{i}^{l+1} = 2(1-\lambda^2)u_i^l + \lambda^2(u_{i+1}^l + u_{i-1}^l) - u_i^{l-1} 
\end{equation}

donde $\lambda \coloneq \frac{\alpha \Delta t}{\Delta x}$

Note que $1-\lambda^2$ dicta el factor de amplificación , de clase se sabe que la magnitud de este valor debe ser menor a uno para que la solución sea estable, además dado que $\lambda^2$ es un número real positivo, se tiene

\begin{equation}
\begin{gathered}
2(1 - \lambda^2 ) \leq 1 \\
1 - \lambda^2 \leq \frac{1}{2}
\end{gathered}
\end{equation}

Suponga que $\lambda > 1$, entonces se tiene que $|1-\lambda^2| > 0 $ lo que implica que el factor de amplificación



\end{document}